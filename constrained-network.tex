	\section{Constrained node network}
	In this little section we will describe what is a constrained device and what is a constrained node network,
	these two concepts are important because they are recurrent in my thesis and without a proper introduction, you may have
	some doubts. \newline
	
	A constrained device is small, has low computational power and little memory, it virtually lacks the resources
	we could expect from a modern Internet node; examples of constrained device are: sensors, dated micro-controller and hardware platform like Arduino.\newline
	While dealing with these devices optimization is crucial on various front: the power consumption is not negligible,
	and the same can be said about the bandwidth usage (if available), even the maximum code complexity and the total RAM available are parameters that we must take into account.\newline
	When a constrained device is attached to a network it is also identified as a constrained node, two or more nodes form a network that may have some constraints such as: unreliability, lossy channel, dynamic topology and unpredictable bandwidth, and these constraints characterize the constrained network.\newline
	In order to be more precise, a network is a constrained-node network when the constraints are imposed by the node
	and not by the network itself.
	
		\begin{table}[h!]
		\begin{center}
			\begin{tabularx}{\textwidth}{|l|l|l|X|}
				\hline
				\textbf{Class} & \textbf{Data size} & \textbf{Code size} & \textbf{Description}\\
				\hline
				C0 & $\ll 10KiB$  & $\ll  100KiB$ &A device of class C0 has no direct access to the Internet,
												   or it may not establish a secure connection.
												   It is pre-configured or configured by an human operator.\\
				\hline
				C1 & $\sim 10KiB$ & $\sim 100Ki$B &A C1 device has no full stack protocol support
													but can handle special purpose protocol,
													memory and CPU must be used carefully.\\
				\hline
				C2 & $\sim 50KiB$ & $\sim 250KiB$ &Less constrained and capable of handling
												   the same protocol stack of a server,
												   but the amount of energy available is limited.\\
   				\hline
			\end{tabularx}
			\caption{Constrained devices classification.}
			\label{tab:table1}
		\end{center}
	\end{table}
	
	A constrained device belongs to one of the classes described in table \ref{tab:table1}.\newline
		\begin{table}[h!]
		\begin{center}
			
			\begin{tabular}{|l|l|}
				\hline
				\textbf{Class} & \textbf{Type of energy limitation}\\
				\hline
				E0 & Event energy limited.\\\hline
				E1 & Period energy limited.\\\hline
				E2 & Lifetime energy limited.\\\hline
				E3 & No direct quantitative limitation to available energy.\\\hline
			\end{tabular}
			\caption{Constrained devices energy classification.}
			\label{tab:table2}
		\end{center}
	\end{table}
	
	Also, a device is classified by its energy consumption as illustrated in table \ref{tab:table2}.
	
	A device is classified as E0 when the energy is consumed only when a certain event occurs, an E1 device has a battery 
	that is periodically recharged or replaced.\newline
	On the other hand an E2 device has a non-replaceable primary battery and when it exhausts, the device life cycle ends; lastly an E3 device has un unlimited source of energy so it is not constrained on this front.\newline 
	
	The main sources of energy consumption are: wireless and long communication; in order to save as much energy as possible different strategies can be applied in various scenarios.\newline
	If there is no need for extreme power saving a device can be always on, in this case it is necessary to use power friendly hardware in order to avoid wastes of energy and money.\newline
	In other cases a device could be off most of time and it will wake up only when needed, in this case the wake up and the reconnection to a network can be costly from an energy point of view, but it is considered a good trade off.\newline
	A device may have low power consumption as it can operate with small amount of power; in this case it is a special case of an always on device.