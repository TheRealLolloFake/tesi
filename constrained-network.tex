	\section{Constrained node network}
	A constrained device is small, has low computational power and little memory.
	Basically, it lacks the resources that we could expect from a basic Internet node.\newline
	Optimization of power consumption and bandwidth is extremely important; even the maximum code complexity and the maximum RAM available must be taken into account.\newline
	When a constrained device is attached to a network it is identified as a constrained node, lots of constrained nodes form a network that may have some constraint such as: unreliability, lossy channel, dynamic topology and unpredictable bandwidth.\newline
	A constrained network is characterized by: low bandwidth, high packet loss and asymmetric link characteristics, absence of multicast communication and high penalties in using large packets.\newline
	A network is called constrained-node network when the constraints are imposed by the node.
	
		\begin{table}[h!]
		\begin{center}
			\begin{tabularx}{\textwidth}{|l|l|l|X|}
				\hline
				\textbf{Class} & \textbf{Data size} & \textbf{Code size} & \textbf{Description}\\
				\hline
				C0 & $\ll 10KiB$  & $\ll  100KiB$ &A device of class C0 has no direct access to the Internet,
												   or it may not establish a secure connection.
												   It is pre-configured or configured by an human operator.\\
				\hline
				C1 & $\sim 10KiB$ & $\sim 100Ki$B &A C1 device has no full stack protocol support
													but can handle special purpose protocol,
													memory and CPU must be used carefully.\\
				\hline
				C2 & $\sim 50KiB$ & $\sim 250KiB$ &Less constrained and capable of handling
												   the same protocol stack of a server,
												   but the amount of energy available is limited.\\
   				\hline
			\end{tabularx}
			\caption{Constrained devices classification.}
			\label{tab:table1}
		\end{center}
	\end{table}
	
	A constrained device belongs to one of classes described in table \ref{tab:table1}.\newline
		\begin{table}[h!]
		\begin{center}
			
			\begin{tabular}{|l|l|}
				\hline
				\textbf{Class} & \textbf{Type of energy limitation}\\
				\hline
				E0 & Event energy limited.\\\hline
				E1 & Period energy limited.\\\hline
				E2 & Lifetime energy limited.\\\hline
				E3 & No direct quantitative limitation to available energy.\\\hline
			\end{tabular}
			\caption{Constrained devices energy classification.}
			\label{tab:table2}
		\end{center}
	\end{table}
	
	A device can also be classified from the point of view of energy consumption as illustrated in table \ref{tab:table2}.
	
	The main sources of energy consumption are: wireless communication and long communication; in order to save as much energy as possible different strategies could be applied in different scenarios.\newline
	If there is no need for extreme power saving a device could be always on, in this case it is necessary to use power friendly hardware in order to avoid wastes of energy and money.\newline
	In other cases a device could be off most of time and it will wake up only when needed, in this case the wake up and the reconnection the network
	could be costly from an energy point of view, but it is considered a good trade off.\newline
	It is also possible that a device has low power consumption as it can operate with small amount of power; in this case it is a special case of an always on device.