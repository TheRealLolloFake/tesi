	\begin{table}[h!]
		\begin{center}
			\begin{tabularx}{\textwidth}{|l|X|}
				\hline
				\textbf{Name} & \textbf{Description}\\\hline
				Content-Format & Specifies the format of the payload.
				If this Option is missing and a payload is included, no default value is assumed.\\\hline
				
				ETag & Specifies an entity-tag, it is a local identifier for the same resource that may vary over time.
				It is generated by the server that provides the resource.
				The entity-tag can be crafted in different way: version, checksum, hash or time.
				The client that receives an ETag must make no assumptions about it.
				If used as a response Option it provides the current value of the entity-tag.
				On the other hand if it is used as a GET request option, it is used to select one or more representation.\\\hline
				
				Location-Path & to do\\\hline
				
				Location-Query & to do\\\hline
				
				Max-Age &	Specifies the maximum time a response may be cached before it is considered not fresh.
				The value is an integer between 0 and $2^32-1$, the default value is 60 seconds.\\\hline
				
				Proxy-Uri & It is used to perform a request to a forward-proxy.
				If an endpoint receives a request with a Proxy-Uri Option and it is not capable to act as forward-proxy it must send back a 5.05 response.
				It must take precedence over any Uri-Host, Uri-Port, Uri-Path or Uri-Query.
				They must not be included in a request that has the Proxy-Uri Option.\\\hline
				
				Proxy-Scheme & to do\\\hline
				
				Uri-Host & Specifies the Internet host of the resource requested.
				The default value is the IP literal of the IP address.\\\hline
				
				Uri-Path & Specifies a segment of the absolute path to the resource.
				It can contain any character sequence expect from “.” and “..”.\\\hline
				
				Uri-Port & Specifies the transport-layer port number of the resource.\\\hline
				
				Uri-Query & Specifies one argument parameterizing the resource.
				It can contain any character sequence.\\\hline
				
				Accept & Specifies which Content-Format is acceptable to the client.
				If this Option is missing then the client does not express a preference.
				If the specified content format is not available the server must send back a 4.06 response.\\\hline
				
			\end{tabularx}
			\caption{Constrained devices classification.}
			\label{tab:table7}
		\end{center}
	\end{table}