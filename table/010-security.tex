	\begin{table}[h!]
		\begin{center}
			\begin{tabularx}{\textwidth}{|l|X|}
				\hline
				NoSec & In this mode there is no security because DTLS is disabled.
				Basically, messages are sent over the normal UDP channel and the coap scheme is used.\\\hline
				PreSharedKey & DTLS is enabled and the device has a list of pre-shared keys that it uses to communicate with the other nodes in the network.
				It is possible to have a one key for each node in the network but it may be not possible for memory constraint.
				It uses the coaps scheme.
				When an endpoint starts a communication with another endpoint an appropriate key is selected.
				The cipher suite must implement \texttt{TLS\_PSK\_WITH\_AES\_128\_CCM\_8}.\\\hline
				RawPublicKey & DTLS is enabled and the device has an asymmetric key but no certificate.
				It uses the coaps scheme.
				The symmetric key is generated and installed by the manufacturer of the device.
				A device may have multiple public keys.
				The cipher suite must implement \texttt{TLS\_ECDHE\_ECDSA\_WITH\_AES\_128\_CCM\_8}.\\\hline
				Certificate & DTLS is enabled and the device has an asymmetric key and a signed certificate.
				The device has a list of root trust that it uses to validate the certificate.
				It uses the coaps scheme.
				The certificate needs to be verified when a new connection is created, if the CoAP node is capable of computing an absolute time, then the node should check the validity of the certificate.
				The cipher suite must implement \texttt{TLS\_ECDHE\_ECDSA\_WITH\_AES\_128\_CCM\_8}.\\\hline
				
				
			\end{tabularx}
			\caption{Security options available in CoAP.}
			\label{tab:table10}
		\end{center}
	\end{table}