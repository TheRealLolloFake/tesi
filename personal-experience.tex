

\chapter{Conclusion}\label{ch:end}

In this last chapter, we recap our impressions on CoAP and WebSocket, in order to draw some conclusions; then, I describe my experience during the internship, which applications I worked on, which problems I noticed, how I fixed them and how I think we can improve the software development process from the security point of view.

\section{Considerations on the protocols}

Even if CoAP, at the moment, is not used in real world applications, we think it could be a good choice for a lots of IoT applications. The fact it is lightweight and connectionless makes it perfect for scenarios where the devices send data at
regular intervals. In these scenarios establishing a connection would be costly and useless, because it would be closed immediately after the data forwarding; on the other hand, mantaining a long lasting connection would drain the battery life of our devices.\\
The fact that CoAP easily translate to HTTP is surely valuable; web services are quite common in the IoT world and having
an easy way to interact with another protocol can speed up the development process.\\
The major drawback of CoAP is the fact that messages could be delivered out of order, but it can be bypassed by adding a 
timestamp to each message.\\
Due to the fact we are dealing with constrained devices, it is important to highlight that it is not obvious that every device
supports CoAP and its four security modes; for instance, a device might support just one or two modes. This drawback is not strictly related to the protocol itself, but we think that finding a good trade-off on the computational power of the device is crucial in order to support CoAP and keep the costs low.\\

WebSocket is quite common and its usage is very simple, due to the API exposed by browsers, which is intuitive.
While it is easy to use, at the same time it is also easy to misuse, and that could introduce flaws in an application.
Enforcing data validation should be always suggested in tutorials and reference manuals but, at the moment, most documentation found in
the Web does not cover the security aspects properly.\\
The only document that clearly enforces data validation is the RFC, but due to its technicality, it is not likely that
a developer, who only uses the protocol (and does not implement it), will read it.\\
Another thing we have noticed that there is no WebSocket server library that stands out from the others, basically there is
not a de facto standard or library that everyone suggests to use.\\
From the client side point of view the problem is the same, but if you use WebSocket as expected, you can use the embedded
protocol implemented by the browser vendors to bypass the problem.



\section{Internship}

During my internship at Abo Data I had the opportunity to work on different real world applications and I found some security problems that I fixed.\newline
In this section, I will not refer the applications with specific names in order to avoid disclosure of information that could damage Abo Data or its customers.\\

Application 1 is an IoT application, which manages railway networks. It currently handles network devices, but in the future it will also support devices like actuators to control the railway.
The application provides a dashboard that allows the user to control network devices: it is possible to read and edit a configuration.
The dashboard is implemented as a web application, in order to make it available even on mobile devices, because the customer needs
to work with this application even on the field.\newline
The application allows to create new networks, to monitor the actual state of the network, to check the last alarms (basically notifications from the network), perform query on the DB, restoring or exporting the DB and managing users (it is possible to create, delete and update a certain user).\\
Even if the application is in advanced state of development, from what I have heard, in future there will be added lots of new features.\\

Application 2 is another IoT application, which deals with fleet of vehicles and their tires. It is a large system developed by Abo Data and another important company.
This system allows to track vehicles and monitor their tires status, in order to provide data to the end user.\newline
The telemetry gathered from a vehicle includes: speed, location, engine parameters, G-force experience during cornering, braking and acceleration.\newline
This kind of information can be used to track the status of a vehicle, in order to verify if everything is working well and to avoid unwanted faults.\newline
All data is collected and stored on clouds and databases, in this way an user can review the data to make decisions; usually to reduce fuel consumption.\newline
This data can be used for a lot of different tasks: the end user can track his fleet of vehicles and monitor if everything is right, while the tire company can verify that the produced tires comply with the standard.\newline
This also allows us to talk about maintenance of vehicles and tires from a different point of view: from a time based
approach, where the vehicle and its tires are controlled every $X$ year, to a predictive based approach where the maintainer acts on the vehicle when the elaborated data allows him to predict a fault or imminent failure.\newline
The system works in this way: there are sensors placed on the vehicle, each sensor communicates (wireless) with a central control unit that collects their data, which is sent via a wireless channel to a remote server for processing.\\

I worked on a sub-system known as Single Sign On (SSO). This sub-system allows users to log into more than one system,
with only one set of credentials, with the help of a token.\newline
When I started my internship this sub-system was not ready, for this reason we focused on making it secure by design; this led us on working a bit more on the analysis of the sub-system, before starting with the real implementation in order to avoid the presence of security flaws that could be difficult to fix in future.\newline


\section{Problems detected in Application 1}
Application 1 had five important flaws:
\begin{enumerate}
	\item It trusted user input
	\item It did not check user privilege
	\item The passwords were stored with symmetric cryptography
	\item Passwords were handled as String
	\item It was vulnerable to DoS
\end{enumerate}

The first flaw allowed a malicious user to log into the system with higher privilege by simply changing a line of code in the JavaScript code on the client side.\newline
The system was designed in this way: the client sent username, password and the privilege level to the server. At the beginning the privilege level was set to 0, but if the login was successful it would replace it with the real privilege level of the user.\newline
The user's privilege level was needed on the client side in order to choose which options to show to the user, based on his actual privileges, so the application did not show all the features to every user, but they were available for use even if hidden.\newline
Of course, by setting the privilege level to the maximum, it was possible to access all the features of the application without restriction.\newline
All the functionalities, reserved to high privilege users, were available to everyone because on the server there was no check.\newline
Checks were on the client side, but by simply changing a line of code it was possible to bypass them, and the server trusted the input because it assumed that the client was not corrupt.\newline
In order to fix this flaw, the login method on the server was modified to load the data from the DB instead of trusting the one provided by the client.\newline

The second flaw is strictly related with the first one: since the server trusted the client, it did not check if the user had enough privilege to perform certain actions.\newline
Basically, the server received a request and performed it without checking who sent it.\newline
In order to fix this flaw it was necessary to add checking on top of each method that performed an action requiring a certain privilege.\newline
One side effect of this problem forced us to re-design the token management of the application because, during the login, the user was associated with the privilege level sent by the client, so adding the check was not enough. In fact if a malicious user logged into the system with an high privilege level, the server would have trusted him and would have associated a token with his fake privilege level.\newline
In order to make it work properly it was necessary to associate the privilege level saved in the database.\newline

The third flaw was the easiest to fix because it required changes only in one part of the code.\newline
The passwords were saved on the DB with the aid of symmetric cryptography; the encryption key was stored in a configuration file.\newline
Saving passwords in this way is not secure because symmetric cryptography is, by design, reversible; a password must be stored in a DB with the aid of a one way function, like a hash function or a key derivation function.\newline
To fix this flaw we replaced the symmetric cryptography algorithm with a key derivation function, the DB required some change in order to store the number of iteration and the salt.\newline

The fourth flaw was easy to fix because it required replacing String with array of chars; this change was needed in order to minimize the exposure of sensitive data and to reduce the success of cold boot attacks.\newline
Fortunately, the amount of code was limited, otherwise changing lots of occurrences would have been a mess.\\

The fifth flaw allowed a malicious user to perform a denial of service by exploiting a bug, which required the attacker to steal a connection
token. This was possible because the communication between client and server was not encrypted.\\
Even if this should suggest a session hijacking vulnerability, in reality it led us to a denial of service.\\
If the attacker stole the connection token of another user, and placed it in the client side code, it was possible
to perform a denial of service attack: the token of the legitimate user was removed from the server, and assigned to the malicious user.
Since the attacker had another associated token, and it is not possible to associate more than one token at the same time, the exception was left unhandled and the system was vulnerable; in order to fix this flaw we handled the exception and avoided the disconnection for the legitimate user.\\

These five flaws showed us that sometimes adding security to an application may be easy, but at the same time very difficult if some components are strictly related each other.\\

The system have a simple mechanism to avoid brute force attack, after three attempts the account under attack is locked and there is no way to access
until it is unlocked by an administrator.\\
From a point of view this seems like a reasonable solution, but take in mind that an attacker could try to lock all the accounts in order to carry out
a DoS (as counter measure there is a special administrator account that can not be locked).\\
Talking about XSS vulnerabilities, it seems like the system is not vulnerable and this is good. The application runs Angular, which is a
framework to develop single page web applications. Even if I do not really like it, because I think it is overly verbose and often makes simple tasks more complicated due to its impositions, it provides a dual-way data binding that takes care of preventing XSS vulnerabilities.\\
I tried several times to inject code, in order to discover vulnerabilities, but Angular always avoided my injections, since Angular
does not allow execution of inline JavaScript code.\\

The application uses WebSocket to communicate with the server, I have noticed that the \texttt{Origin} header is not used and neither the \texttt{Content-Security-Policy}. I was not able to exploit these little vulnerabilities, but it must be considered as a warning that something
should be fixed.\\
In the debug version, communications occur via an unencrypted channel, this means that a MitM can occur; on the other hand,
the deployed version uses HTTPS and secure WebSocket; this mean that the fifth flaw was easy to exploit only in the debug version.

\section{Design of Application 2}
As said, for Application 2 we focused more on the design phase: at the beginning the idea was to
store passwords with symmetric cryptography because there were no alternatives (as we know from the best practices, it is not a good idea to store password with this kind of cryptography), the external systems are not under our control so
it was not possible to use tools like OAuth.\\
Fortunately in the back-end of the application there was a function that allowed us to avoid to store the external systems' passwords,
for this reason our work focused on creating secure tokens, the tokens used in SSO are signed and an attacker cannot forge a fake one
without knowing the signing key.\\
The token is generated with the aid of a secure pseudo-random number generator and other data provided by the user: they are mixed in order to
create an array of bytes which will be the argument of the HMAC-SHA1 algorithm.\\
Unfortunately the key is stored in a plain text configuration file, we discussed how to store it in a secure way but this was the only feasible
solution; an HSM was not available, and storing the key in another machine was not feasible because it required a certificate in order to
secure the communication between the two servers.\\


\section{Analysis of the security aspect and conclusion}
During the internship I noticed that the security aspects, during the development of an application, are not as easy to meet as it may seem.\newline
From an academic point of view, making a secure application is the main goal, but from a business perspective, such a goal is not so easy to reach because there are other factors that must be kept in mind.\newline
If you are developing an application by yourself and for yourself, you have no deadlines, no customers to meet and no budget limitations.\newline
On the other hand, if you are working in a company, you are developing an application for a customer and you must take care of the:
\begin{itemize}
	\item Relationship with him
	\item Deadlines
	\item Support for the application after the release
\end{itemize}

In order to maintain a solid relationship with your costumer, deadlines must be respected, and here the development of the application comes to play; including security in the development of an application may seems easy at the beginning, but it is not so simple and it may cost a lot sometime. When there is a deadline, the goal is to finish the job no matter what.\newline
This is one reason why some developers simply avoid security aspects during the development of an application.\newline

Another reason is about money: some customers are not interested or not even aware about security issues that an application could have, so they are simply not interested in paying for it, but when a security problem occurs they will surely come yelling at you.\newline
In this case, I think that if you could add security to an application and the time spent is equal to the time spent in making the same thing in an insecure way, then I suggest you to go for the secure way.\newline
Of course if something requires lots of time, talk about it with the customer and convince him to pay for the additional security features.\newline
Usually, when company A sells an application to company B, company A provides technical support for a certain period of time; this means that company A must adjust the software if something is not working properly, and it should provide updates.
If there is no time to add a certain feature on the first release, it could be added later with an update.\newline

During the development of software, it is common that the development team adopts certain software development process; Abo Data adopted an agile software development process.\newline
From what I have noticed it is the right choice because the application requirements are changing frequently, but from the security point of view I think it is a bit problematic and could lead to flaws in the code.\newline
The security part on an application should be defined in a way that allows the development team to edit it at least as possible.\newline

The security part should also be developed with particular care in order to be easily updated in the future; suppose that the hashing algorithm employed in your application becomes insecure for some reasons, then you should update your application, but of course an update has a cost and requires a certain amount of time in order to be completed. If the code is monolithic it will be difficult and expensive to update, on the other hand if the code is modular it will be simpler.\newline
Another thing to take in mind is that in the IoT world we also deal with constrained devices, if for some reason you need to change the algorithm to a new one, then you should check if your constrained devices are capable of handling the new algorithm, otherwise you should stick with the old algorithm or use more powerful devices. Of course, you have to find a trade-off.\newline
From my experience I noticed that implementing a security feature or fixing security flaws is difficult in software that is in an advanced stage of development, it may require substantial changes to the whole project, and this could slow down the development and it could be difficult to meet the deadline.\newline
Basically, security should be part of the application since the requirements phase, or big problems will emerge later on.\newline

In order to produce a secure software we must focus on two aspects: the development process and software security testing.\\
Talking about the development process, there are established methodologies like the Security Development Lifecycle (SDL) proposed by Microsoft~\cite{microsoftsdl} which provide a standard approach for security development.\\
SDL introduces some stages in the classic software development life cycles which consists of:
\begin{itemize}
	\item Requirements
	\item Design
	\item Implementation
	\item Verification
	\item Release
	\item Response
\end{itemize}
Of course, the stages we will discuss soon can be added in different software development methodologies without problems.\\
During the requirements phase, the best practices for security are integrated, they come from industry standard or from
solution adopted on responses to problems addressed in the past.
In this phase the functional security requirements are defined in order to be implemented in the final software, 
these requirements are exactly the same used to define the features of an application, they only focus on the security front.\\
If the methodology is agile, then the requirements are expressed as user stories, in this case these stories concern the security aspects from the user's point of view.\\

During the design phase a threat model is defined in order to define how a system will be attacked. Of course, this
activity must occur before the implementation, otherwise it will not be effective.\\
The threat model is used to understand the attack surface of a certain features and what are the common attacks that
the development team must face.\\
It is also important to define a way to mitigate the problem if a certain attack occurs, and it is crucial to identify
security issues early on.\\

The implementation phase is surely the most difficult one because writing secure code is difficult, and even if you have
a security team that take care of the most important part of the application, nothing can prevent a normal programmer to introduce a security bug in a part of the application that is considered not critical.\\
It is important to define coding guideline in a similar way to best practices in order to avoid mistakes by programmers.\\
It is also important to use tools for static and dynamic application security testing (SAST and DAST), the first one is used to
identify potential vulnerabilities in the source code, while DAST checks the application at runtime.\\

During the testing phase, in addition to the standard test on the application, we should perform other activities like:
security functional testing, vulnerability scanning and penetration testing.\\ 

When the software is finally released, it is shipped to customers or made available for download; of course there will be
problems, but if everything was properly done they should have been minimized.\\
During the final response phase, the customer will notify problems occurred with the application in order to receive
an update that fixes them.\\

About security testing there is a lot to say because it covers different activities, but we just give you a small overview:
\begin{itemize}
	\item Discovery - this activity does not search flaws in the code, but it only focuses on identifying the version of software in use, this is important because by the version number we can discover potential vulnerabilities.
	\item Vulnerability scan - this activity is carried out with automated tools that search for
	known vulnerabilities in the application.
	\item Penetration test - it is a simulation of an attack, it tries to exploit flaws in the application like a real attacker would.
	\item Security Review - verifies if security standards have been applied to the system; it is
	done via analysis of code and different versions of the program.
\end{itemize}


Security testing can be conducted in a black box way, where the tester does not know details of the system and the architecture behind it; or
in a white box settings where the tester has deep knowledge of the system and he may have access to the source code.\\
Due to the fact that each application is different, the security testing technique should adapt to the application taken under examination in order
to be as accurate as possible; for instance if you are testing a web application client side, it is not likely you will find a buffer overflow, but 
an XSS is more likely.\\

Of course, security testing is quite more difficult compared to standard testing (that is, by itself, a difficult task), for this reason companies prefer to avoid it or pay third parties that are specialized in this particular activity.\\
Take also in mind that security testing may not provide you all the vulnerabilities in your system, but only a certain amount; for this reason it
is important to take care about your software periodically; when you change a part of your application, you should check that changes do not
introduce flaws.\\

Another aspect of security, we did not mention in this thesis, is privacy; even if security and privacy are two distinct concepts, they are related because without security there is no privacy, I think it is important to discuss it a bit due to the fact that the General Data Protection Regulation (GDPR)
has been recently approved by the European commission, in order to strength the protection of personal data of citizens of the European Union.\\
This new regulation should allow citizens to have more control over their data and it should also make sure that companies implement secure system
in order to avoid data breaches, considering that there are fines if a company does not notify the supervisory authority within 72 hours after becoming aware of the leak; this is also enforced by Article 25 that requires data protection to be designed into the development of business processes~\cite{wikica}.\\
The GDPR takes also into consideration the trade of data between companies, an hot topic due to recent events~\cite{ca}.\\


We would like to highlight that even documentation and refactoring are important while dealing with security; most of the time documentation is neglected because it takes lot of time and is not paid by the customer. Working on a big project for long time
requires good documentation to understand the purpose of certain code and avoid the introduction of flaws.\\
Talking about refactoring, it is extremely important because it reduces the amount of lines of code and
avoids absurd copy-and-paste of code; suppose that a particular piece of code is vulnerable and it is executed
before every method in your code, if you refactored your code well, then you will need to fix your code only once, otherwise
if you applied the copy-and-paste ``technique'' you will need to fix the code before every function, and it is likely that you will forget some occurrences and the flaw will remain in your code; for this reason refactoring is extremely important.\\
Thus even the maintenance of software plays an important role on its security and this should make us consider that every
aspects in software development, even little details, should be treated with particular care in order to produce secure software.
